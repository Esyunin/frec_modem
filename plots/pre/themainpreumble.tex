% Класс документа пока не окончательный, сильно сомневаюсь, что article лучший
\documentclass[a4paper,12pt]{article} 
% Подключаем шрифты,кодировки,русские переносы
\usepackage{cmap}
% подключается пакет, позволяющий улучшить вид пдф документа(как я понял)
\usepackage[T2A]{fontenc}
\usepackage[utf8x]{inputenc}
% подключаем кодировку шрифтов для вносимых файлов
\usepackage[main=russian,english]{babel}
% подключаем перенос и распознование слов, русский в приоритете
\usepackage{indentfirst}
% Отступ в начале абзаца
\usepackage{
	amssymb,
	amsfonts,
	amsmath,
}
% Пакеты американского математ. сообщества, красивый вид формул и текста внутри
\usepackage{
	wrapfig,
	graphicx,
	caption,
	subcaption,
	tikz,
}
% Обтекаемые объекты, рисунки, подписи и прочее
\usepackage{
	pgfplotstable,
	pgfplots,
	booktabs,
	colortbl,
	array
}
\pgfplotsset{compat=newest}
% таблицы, графики

% \usepackage{xcolor}
\usepackage[unicode]{hyperref}

 % Цвета для гиперссылок
\definecolor{linkcolor}{HTML}{000000} % цвет ссылок
\definecolor{urlcolor}{HTML}{799B03} % цвет гиперссылок
\hypersetup{pdfstartview=FitH,  linkcolor=linkcolor,urlcolor=urlcolor, colorlinks=true}

\usepackage{geometry}
\usepackage{fancyhdr}
% границы, контитулы, и прочее


\geometry
	{
	left=2.2cm,
	right=2.2cm,
	bottom=2cm,
	top=2cm,
	}
% границы документа

\usepackage{setspace}
% убирает гигантские размеры оглавления
\linespread{1.3}
% междустрочный интервал

\pagestyle{fancy}
\fancyhead{}
% пустая шапка контитула
\fancyhead[R]{\authors}
% На правой стороне страницы авторы и науч.рук.
\fancyhead[L]{\shortlabname}
 % Слева название лабы
\fancyfoot{}
\fancyfoot[C]{\thepage}
% номер страницы снизу по середине
\renewcommand{\contentsname}{Оглавление}
% переводим на русский язык оглавление
\usepackage{secdot}
\sectiondot{subsection}
% Ставит злосчастные точки в главах, ибо не по госту
% Преамбула почти слизана у Федора Сарафанова https://github.com/FedorSarafanov/RLC/blob/master/text/diss.tex
